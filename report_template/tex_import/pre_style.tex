%%%%%%%%%%%%%%%%%%% anything related to better presentation %%%%%%%%%%%%%%%%%%%
%%%%%%%%%%%%%%%%%%% configure tableofcontents 
\AtBeginSection[]
{
    \begin{frame}
        \frametitle{Table of Contents}
        \tableofcontents[
            currentsection,
            currentsubsection,
            subsectionstyle=show/shaded/hide
        ]
    \end{frame}
}
\AtBeginSubsection[]
{
    \begin{frame}
        \frametitle{Table of Contents}
        \tableofcontents[currentsection,currentsubsection]
    \end{frame}
}

%%%%%%%%%%%%%%%%%%% double-screen presentation
\usepackage{pgfpages}
% These slides also contain speaker notes. You can print just the slides,
% just the notes, or both, depending on the setting below. Comment out the want
% you want.

\setbeameroption{hide notes} % Only slides
% \setbeameroption{show only notes} % Only notes
% \setbeameroption{show notes on second screen=right} % Both

% To give a presentation with the Skim reader (http://skim-app.sourceforge.net) on OSX so
% that you see the notes on your laptop and the slides on the projector, do the following:
%
% 1. Generate just the presentation (hide notes) and save to slides.pdf
% 2. Generate only the notes (show only nodes) and save to notes.pdf
% 3. With Skim open both slides.pdf and notes.pdf
% 4. Click on slides.pdf to bring it to the front.
% 5. In Skim, under "View -> Presentation Option -> Synchronized Noted Document"
%    select notes.pdf.
% 6. Now as you move around in slides.pdf the notes.pdf file  will follow you.
% 7. Arrange windows so that notess.pdf is in full-screen mode on your laptop
%    and slides.pdf is in presentation mode on the projector.

% Give a slight yellow tint to the notes page
\setbeamertemplate{note page}{\pagecolor{yellow!5}\insertnote}
\setbeamerfont{note page}{size=\large}
\usepackage{palatino}

% configure beamer template.
\setbeamertemplate{theorems}[numbered]
\usefonttheme[onlymath]{serif}

% configure the margin for itemize.
\settowidth{\leftmargini}{\usebeamertemplate{itemize item}}
\addtolength{\leftmargini}{\labelsep}

%%%%%%%%%%%%%%%%%%% highlight something.
\usepackage{colortbl}
\makeatletter
\let\@@magyar@captionfix\relax
\makeatother

%%%%%%%%%%%%%%%%%%% highlight something, v2
\colorlet{lightGreen}{green!50!white}
\colorlet{lightTeal}{teal!50!white}
\colorlet{lightRed}{red!50!white}
\newcommand{\highlightt}[2][yellow]{\mathchoice%
    {\colorbox{#1}{$\displaystyle#2$}}%
    {\colorbox{#1}{$\textstyle#2$}}%
    {\colorbox{#1}{$\scriptstyle#2$}}%
    {\colorbox{#1}{$\scriptscriptstyle#2$}}}%


%%%%%%%%%%%%%%%%%%% for footnote-like bib.
\newcommand\blfootnote[1]{%
    \begingroup
    \renewcommand\thefootnote{}\footnote{#1}%
    \addtocounter{footnote}{-1}%
    \endgroup
}


%%%%%%%%%%%%%%%%%%% The table highlighting for hypothesis discussion.
\usepackage[beamer,customcolors]{hf-tikz}
\usetikzlibrary{calc,arrows,shapes,positioning}
\usetikzlibrary{overlay-beamer-styles}

% To set the hypothesis highlighting boxes red.
\tikzset{hl/.style={
            set fill color=red!80!black!40,
            set border color=red!80!black,
        },
}
\newcommand{\tikznode}[2]{%
    \ifmmode%
        \tikz[remember picture,baseline=(#1.base),inner sep=0pt] \node (#1) {$#2$};%
    \else
        \tikz[remember picture,baseline=(#1.base),inner sep=0pt] \node (#1) {#2};%
    \fi}

\tikzset{
    ncbar angle/.initial=-90,
    ncbar/.style={
            to path=(\tikztostart)
            -- ($(\tikztostart)!#1!\pgfkeysvalueof{/tikz/ncbar angle}:(\tikztotarget)$)
            -- ($(\tikztotarget)!($(\tikztostart)!#1!\pgfkeysvalueof{/tikz/ncbar angle}:(\tikztotarget)$)!\pgfkeysvalueof{/tikz/ncbar angle}:(\tikztostart)$)
            -- (\tikztotarget)
        },
    ncbar/.default=0.5cm
}
\tikzset{
    hyperlink node/.style={
            alias=sourcenode,
            append after command={
                    let     \p1 = (sourcenode.north west),
                    \p2=(sourcenode.south east),
                    \n1={\x2-\x1},
                    \n2={\y1-\y2} in
                    node [inner sep=0pt, outer sep=0pt,anchor=north west,at=(\p1)] {\hyperlink{#1}{\XeTeXLinkBox{\phantom{\rule{\n1}{\n2}}}}}
                    %xelatex needs \XeTeXLinkBox, won't create a link unless it
                    %finds text --- rules don't work without \XeTeXLinkBox.
                    %Still builds correctly with pdflatex and lualatex
                }
        }
}


%%%%%%%%%%%%%%%%%%% for tcolorbox
\usepackage{tcolorbox}
\tcbuselibrary{skins}
\usepackage{bbding}
\usepackage{array}
\usepackage{arydshln}

% Define a custom tcolorbox environment for takeaway messages
\tcbset{
    takeaway/.style={
            colback=blue!10!white, % Background color
            colframe=blue!75!black, % Frame color
            coltitle=black, % Title color
            boxrule=0.5mm, % Thickness of the frame
            arc=3mm, % Arc size of the box corners
            outer arc=0mm, % Outer arc
            fonttitle=\bfseries, % Title font
            title={Takeaway Message}, % Default title
            attach boxed title to top center={
                    yshift=-2mm,
                    yshifttext=-1mm
                },
            boxed title style={
                    size=small,
                    colback=blue!75!black,
                    before upper={\rule[-3pt]{0pt}{16pt}}, % vertical centering
                    colframe=blue!75!black,
                },
            enhanced,
        }
}